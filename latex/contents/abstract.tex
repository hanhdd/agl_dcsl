\begin{abstract}
Domain-driven design (DDD) aims to iteratively develop software around a realistic domain model, i.e., that can be viewed as a program from the developer's perspective. Recent works in DDD have been focusing on using annotation-based domain-specific languages (aDSLs) to build the domain model. However, within these works behavioral aspects, that are often represented using UML activity diagrams and statecharts, are not explicitly captured as part of the domain model. This approach requires an additional and also challenging  effort to integrate them into the domain model for a realistic one. In this paper, we propose a novel unified domain modeling method to tackle this. We employ our previously-developed aDSL, named \dcsl, in order to express our unified domain model, that includes new domain classes extracted from the underlying behavioral models within the context of a module-based software architecture (MOSA). We then define a novel aDSL, named activity graph language (\agl), to capture the remaining behavioral aspects. The mapping between the two aDSL models allows us to obtain a unified and realistic domain model as an incoporation of the behavioral models and the original domain model. To realize our method, we define a compact annotation-based syntax meta-model and then a formal semantics for \agl. We demonstrate our method with an implementation in a Java framework named \jdomainapp~and evaluate \agl~to show that it is essentially expressive and usable for real-world software.
\end{abstract}