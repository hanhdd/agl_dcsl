\begin{abstract}
%Domain-driven design (DDD) aims to iteratively develop software around a realistic domain model, i.e., that can be viewed as a program from the developer's perspective. Recent works in DDD have been focusing on using annotation-based domain-specific languages (aDSLs) to build the domain model. However, within these works the behavioral aspects, often represented using UML activity diagrams and statecharts, are not explicitly captured in the domain model. A main challenge with this is how to integrate the behavioral aspects into the domain model following the DDD approach. In this paper, we propose a novel unified domain modeling method to tackle this. We employ our previously-developed aDSL, named \dcsl, in order to express the unified domain model. We integrate \dcsl with a new aDSL, named activity graph language (\agl), to capture the behavioral aspects. These aspects are conceputally represented by UML activity diagram, in the context of a module-based DDD software architecture.
%%The integration of the two aDSL models allows us to obtain a unified and realistic domain model as an incoporation of the behavioral models and the original domain model. 
%%To realize our method, 
%We define a compact annotation-based syntax meta-model and a formal semantics for \agl. We demonstrate our method with an implementation in a Java framework named \jdomainapp~and evaluate \agl~to show that it is essentially expressive and usable for real-world software.

\noindent\textit{Context:}
Domain-driven design (DDD) aims to iteratively develop software around a realistic domain model, i.e., that can be viewed as a program from the developer's perspective. Recent works in DDD have been focusing on using annotation-based domain-specific languages (aDSLs) to build the domain model. However, within these works the behavioral aspects, that are often represented using UML activity diagrams and statecharts, are not explicitly captured in the domain model. A main challenge with this is how to integrate the behavioral aspects into the domain model following the DDD approach. 

\noindent\textit{Objective:}
In this paper, we concentrate on defining a novel unified domain modeling method to tackle this. We aim to extend our previously-developed aDSL, named \dcsl, that currently only supports structural aspects with class domain models, by incorporating it with a new aDSL, named activity graph language (\agl), in order to capture behavioral aspects of the underlying domain. These aspects could be conceptually represented by UML activity diagram, in the context of a module-based DDD software architecture.

\noindent\textit{Method:}
Our method consists in constructing a configured unified domain model in the MOSA architecture. We used the annotation attachment feature of the host OOPL to attach an \agl's activity graph directly to the activity class of the unified model, thereby, creating a configured unified model. We adopt the meta-modeling approach for DSLs and employ UML/OCL to specify the abstract and concrete syntax models of \agl: A compact annotation-based syntax meta-model and a formal semantics for \agl is defined. We demonstrate our method with an implementation in a Java framework named \jdomainapp~and evaluate \agl~using a case study to show that it is essentially expressive and usable for real-world software. 

\noindent\textit{Results:}
This work brings out (1) an aDSL (named \agl) to express the domain behaviors for a unified model; (2) a method to incorporate behavior aspects into a domain model by defining precisely a unified model in the context of MOSA; and (3) a unified modeling method for developing object-oriented domain-driven software.

\noindent\textit{Conclusion:}
Our method significantly extends the state-of-the-art in DDD in two important fronts: bridging the gaps between model and code and constructing a unified domain model for both structural and behavioral aspects of the domain. Our proposed aDSLs are horizontal DSLs which can be used to support different real-world software domains.

\end{abstract}