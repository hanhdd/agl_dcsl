\begin{abstract}
Domain-driven design (DDD) aims to iteratively develop software around a realistic domain model, i.e., that can be viewed as a program from the developer's perspective. Recent works in DDD have been focusing on using annotation-based domain-specific languages (aDSLs) to build the domain model. However, within these works the behavioral aspects, often represented using UML activity diagrams and statecharts, are not explicitly captured in the domain model. A main challenge with this is how to integrate the behavioral aspects into the domain model following the DDD approach. In this paper, we propose a novel unified domain modeling method to tackle this. We employ our previously-developed aDSL, named \dcsl, in order to express the unified domain model. We integrate \dcsl with a new aDSL, named activity graph language (\agl), to capture the behavioral aspects. These aspects are conceputally represented by UML activity diagram, in the context of a module-based DDD software architecture.
%The integration of the two aDSL models allows us to obtain a unified and realistic domain model as an incoporation of the behavioral models and the original domain model. 
%To realize our method, 
We define a compact annotation-based syntax meta-model and a formal semantics for \agl. We demonstrate our method with an implementation in a Java framework named \jdomainapp~and evaluate \agl~to show that it is essentially expressive and usable for real-world software.
\end{abstract}