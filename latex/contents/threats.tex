%\newpage
\section{Threats to Validity} \label{sect:threats}
This section discusses theats to validity of both our proposed method and the evaluation method.

\subsection{The Proposed Method}

\textbf{\textit{Integration into a software development process}} is essential for the dissemination of our method in practice. We argue that our method is particularly suited for integration into iterative \cite{larman_applying_2004} and agile~\cite{beck_manifesto_2017} development processes. In particular, the development team (which includes domain experts and developers) would use our tool to work together on developing the configured unified model in an incremental fashion: the developers use \dcsl~and \agl~to create/update the configured unified model and then generate the software from this model. The domain experts give feedback for the model via the software GUI and the update cycle continues. The generated software prototypes can be used as the intermediate releases for the final software.

Further, in both processes, tools and techniques from \abbrv{model-driven software engineering}{MDSE} would be applied to enhance productivity and tackle platform variability. In particular, we would apply PIM-to-PSM model transformation~\cite{kent_model_2002,brambilla_model-driven_2012} to automatically generate our configured unified model from a high-level one that is constructed using a combination of UML class and activity diagrams.

\textbf{\textit{The usability of the software GUI}}, from the domain expert's viewpoint, plays a role in the usability of our method. Although in this paper we did not discuss this issue, we would argue in favor of two aspects of the software GUI, namely simplicity and consistency, which contribute towards its learnability~\cite{folmer_architecting_2004}. Our plan is to fully evaluate GUI usability in future work. First, the GUI design is simple because, as discussed in~\cite{le_domain_2018}, it directly reflects the domain class structure. Clearly, this is the most basic representation of the domain model. Second, the GUI is consistent in its presentation of the module view and the handling of the user actions performed on it. Consistent presentation is due to the application of the reflective layout to the views of all modules. Consistent handling is due to the fact that a common set of module actions (see Section~\ref{sect:actSemantics}) are made available on the module view.

\subsection{Evaluation Method}
%
%\subsection{Discussion} \label{sect:eval-discussion} %
%We present below a number of remarks both generally about \agl~and our method and specifically about our evaluation.

\textbf{\textit{The composition of the configured unified model}} in terms of the unified model and an activity graph model (see Section~\ref{sect:agl}) follows a language composition approach described by Kleppe~\cite{kleppe_software_2008}. In this approach, the composition is formed by language referencing. That is, one component language (called \textit{active language}) references the elements of the other component language (called the \textit{passive language}). In our method, \agl~is the active language and \dcsl~is the passive one.

\textbf{\textit{The evolution of languages}} (including both \agl~and \dcsl) is inevitable if we are to support 
more expressive domain modeling requirements. We discuss in~\cite{le_domain_2018} how \dcsl~is currently expressive only \wrt an essential set of domain requirements that are found to commonly shape the domain class design. We argue that \dcsl~would evolve to support other structural features. For \agl, its ASM would be extended to support other activity modeling features, such as activity group (\S{15.6}~\cite{omg_unified_2015}).

\textbf{\textit{The selection of the unified modeling patterns}} used in our expressiveness evaluation is based on the UML class and activity modeling languages that we currently use to construct the configured unified model. A question then arises as to the adaptability of our method to other behavioral modeling languages (\eg state machine and sequence diagram). We plan to investigate this as part of future work.
