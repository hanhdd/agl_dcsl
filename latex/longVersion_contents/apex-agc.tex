%
\section{Activity graph configuration of \orderman - AGC} \label{apex:agl-agc}
%
\begin{lstcodeplainssm}{The activity class \clazz{HandleOrder} in Java is attached with annotation \textbf{@AGraph}, \textbf{@ANode} and \textbf{@MAct}}{lst:handleOrderAGC}
/**Activity graph configuration */
@AGraph(nodes={
	/* 1 */    
	@ANode(label="1:CustOrder", zone="top",init=true, 
	refCls=CustOrder.class, serviceCls=DataController.class, 
	outNodes= {"2:AcceptOrNot"},
	actSeq={
		// create new and wait until a new object is created
		@MAct(actName=newObject, endStates={Created})
	}),
	/* 2: decisional */    
	@ANode(label="2:AcceptOrNot", 
	refCls=AcceptOrNot.class, nodeType=Decision, 
	outNodes= {"3:FillOrder", "4:EndOrder"}),
	/* 3 */    
	@ANode(label="3:FillOrder", zone="top", refCls=FillOrder.class, serviceCls=DataController.class,
	outNodes= {"5:CustOrder"},
	nodeType=Coordinator,
	actSeq = {
		@MAct(actName=newObject, endStates={NewObject}),
		@MAct(actName=setDataFieldValues, attribNames={"receivedOrder"}),
		@MAct(actName=createObject, endStates={Created})
	}),
	/* 4 */    
	@ANode(label="4:EndOrder", refCls=EndOrder.class, 
	nodeType=Merge, outNodes= {"7:CustOrder"}),
	/* 5 */    
	@ANode(label="5:CustOrder", zone="3:FillOrder",refCls=CustOrder.class, serviceCls=DataController.class,
	outNodes = {"6:Delivery"},
	actSeq = {
		@MAct(actName=showObject, endStates = {Updated})
	}),
	/* 6: Fork */    
	@ANode(label="6:Delivery", zone="3:FillOrder",refCls=Delivery.class, 
	nodeType=Fork, outNodes= {"8:CollectPayment", "9:ShipOrder"},
	actSeq = {
		@MAct(actName=newObject, endStates={NewObject}),
		@MAct(actName=createObject, endStates={Created})      
	}
	),
	/* 7 */    
	@ANode(label="7:CustOrder",zone="6:Delivery", refCls=CustOrder.class, serviceCls=DataController.class, 
	actSeq={
		@MAct(actName=showObject),
		@MAct(actName=updateObject, endStates = {Updated})
	}),
	/* 8 */    
	@ANode(label="8:CollectPayment", zone="6:Delivery",refCls=CollectPayment.class, 
	serviceCls=DataController.class,
	nodeType=Coordinator,
	outNodes= {"10:Invoice"},
	actSeq={
		@MAct(actName=newObject, endStates={NewObject}),
		@MAct(actName=setDataFieldValues, attribNames = {"receivedOrder"}),
		@MAct(actName=createObject, endStates={Created})
	}),
	/* 9 */    
	@ANode(label="9:ShipOrder", zone="6:Delivery",refCls=ShipOrder.class, serviceCls=DataController.class,
	nodeType=Coordinator,
	outNodes= {"12:Shipment"},
	actSeq={
		@MAct(actName=newObject, endStates={NewObject}),
		@MAct(actName=setDataFieldValues, attribNames = {"receivedOrder"}, endStates={Created})
	}),
	/* 10 */    
	@ANode(label="10:Invoice", zone="8:CollectPayment",refCls=Invoice.class, serviceCls=DataController.class, 
	outNodes = {"11:AcceptPayment"},
	actSeq={
		// create new and wait until a new object is created
		@MAct(actName=newObject, endStates={NewObject}),
		@MAct(actName=setDataFieldValues, attribNames = {"order"}, endStates={Created})
	}),
	/* 11 */    
	@ANode(label="11:AcceptPayment", zone="8:CollectPayment",refCls=AcceptPayment.class, serviceCls=DataController.class,
	outNodes= {"14:Payment"},
	nodeType=Coordinator,      
	actSeq={
		@MAct(actName=newObject, endStates={NewObject}),
		@MAct(actName=setDataFieldValues, attribNames = {"invoice"}),
		@MAct(actName=createObject, endStates={Created})
	}),
	/* 12 */    
	@ANode(label="12:Shipment", zone="9:ShipOrder",refCls=Shipment.class, serviceCls=DataController.class,
	outNodes = { "13:CompleteOrder" },
	actSeq={
		// create new and wait until a new object is created
		@MAct(actName=newObject, endStates={NewObject}),
		@MAct(actName=setDataFieldValues, attribNames = {"order"}, endStates={Created})
	}),
	/* 13 */    
	@ANode(label="13:CompleteOrder",refCls=CompleteOrder.class, nodeType=Join, 
	outNodes= {"4:EndOrder"}),
	/* 14 */    
	@ANode(label="14:Payment", zone="11:AcceptPayment",refCls=Payment.class, serviceCls=DataController.class, 
	outNodes= { "15:CustOrder" },
	actSeq={
		@MAct(actName=newObject, endStates = {NewObject}),
		@MAct(actName=setDataFieldValues, attribNames = {"invoice"}, endStates = {Created}),
	}),
	/* 15 */    
	@ANode(label="15:CustOrder", zone="11:AcceptPayment",refCls=CustOrder.class,serviceCls=DataController.class,
	outNodes= { "13:CompleteOrder" },
	actSeq={
		@MAct(actName=filterInput, filterType=FilterCustOrderFromPayment.class),
		@MAct(actName=showObject, endStates = {Updated}),
	}),
})
/**END: activity graph configuration */
%
%
\end{lstcodeplainssm}
%%%%%%%%%%%%%%
\section{Activity class \clazz{HandleOrder}} 
\label{apex:agl-classOrderMan}
\begin{lstcodeplainssm}{The activity class \clazz{HandleOrder} in Java}{lst:handleOrderClsModel}
package org.jda.example.orderman.modules.handleorder.model;
import static jda.modules.mbsl.model.graph.NodeType.Coordinator;
import static jda.modules.mbsl.model.graph.NodeType.Decision;
import static jda.modules.mbsl.model.graph.NodeType.Fork;
import static jda.modules.mbsl.model.graph.NodeType.Join;
import static jda.modules.mbsl.model.graph.NodeType.Merge;
import static jda.mosa.controller.assets.util.AppState.Created;
import static jda.mosa.controller.assets.util.AppState.NewObject;
import static jda.mosa.controller.assets.util.AppState.Updated;
import static jda.mosa.controller.assets.util.MethodName.createObject;
import static jda.mosa.controller.assets.util.MethodName.filterInput;
import static jda.mosa.controller.assets.util.MethodName.newObject;
import static jda.mosa.controller.assets.util.MethodName.setDataFieldValues;
import static jda.mosa.controller.assets.util.MethodName.showObject;
import static jda.mosa.controller.assets.util.MethodName.updateObject;
import java.util.Collection;
import org.jda.example.orderman.modules.delivery.model.CollectPayment;
import org.jda.example.orderman.modules.delivery.model.Delivery;
import org.jda.example.orderman.modules.fillorder.model.FillOrder;
import org.jda.example.orderman.modules.handleorder.control.model.AcceptOrNot;
import org.jda.example.orderman.modules.handleorder.control.model.CompleteOrder;
import org.jda.example.orderman.modules.handleorder.control.model.EndOrder;
import org.jda.example.orderman.modules.invoice.model.Invoice;
import org.jda.example.orderman.modules.order.filter.FilterCustOrderFromPayment;
import org.jda.example.orderman.modules.order.model.CustOrder;
import org.jda.example.orderman.modules.payment.model.AcceptPayment;
import org.jda.example.orderman.modules.payment.model.Payment;
import org.jda.example.orderman.modules.ship.model.ShipOrder;
import org.jda.example.orderman.modules.ship.model.Shipment;
import jda.modules.dcsl.syntax.DAssoc;
import jda.modules.dcsl.syntax.DAssoc.AssocEndType;
import jda.modules.dcsl.syntax.DAssoc.AssocType;
import jda.modules.dcsl.syntax.DAssoc.Associate;
import jda.modules.dcsl.syntax.DAttr;
import jda.modules.dcsl.syntax.DAttr.Type;
import jda.modules.dcsl.syntax.DCSLConstants;
import jda.modules.dcsl.syntax.DClass;
import jda.modules.dcsl.syntax.Select;
import jda.modules.mbsl.model.appmodules.meta.MAct;
import jda.modules.mbsl.model.graph.meta.AGraph;
import jda.modules.mbsl.model.graph.meta.ANode;
import jda.mosa.controller.ControllerBasic.DataController;
/**Activity graph configuration */
@AGraph(nodes={
	/* 1 */    
	@ANode(label="1:CustOrder", zone="top",init=true, 
	refCls=CustOrder.class, serviceCls=DataController.class, 
	outNodes= {"2:AcceptOrNot"},
	actSeq={
		// create new and wait until a new object is created
		@MAct(actName=newObject, endStates={Created})
	}),
	/* 2: decisional */    
	@ANode(label="2:AcceptOrNot", 
	refCls=AcceptOrNot.class, nodeType=Decision, 
	outNodes= {"3:FillOrder", "4:EndOrder"}),
	/* 3 */    
	@ANode(label="3:FillOrder", zone="top", refCls=FillOrder.class, serviceCls=DataController.class,
	outNodes= {"5:CustOrder"},
	nodeType=Coordinator,
	actSeq = {
		@MAct(actName=newObject, endStates={NewObject}),
		@MAct(actName=setDataFieldValues, attribNames={"receivedOrder"}),
		@MAct(actName=createObject, endStates={Created})
	}),
	/* 4 */    
	@ANode(label="4:EndOrder", refCls=EndOrder.class, 
	nodeType=Merge, outNodes= {"7:CustOrder"}),
	/* 5 */    
	@ANode(label="5:CustOrder", zone="3:FillOrder",refCls=CustOrder.class, serviceCls=DataController.class,
	outNodes = {"6:Delivery"},
	actSeq = {
		@MAct(actName=showObject, endStates = {Updated})
	}),
	/* 6: Fork */    
	@ANode(label="6:Delivery", zone="3:FillOrder",refCls=Delivery.class, 
	nodeType=Fork, outNodes= {"8:CollectPayment", "9:ShipOrder"},
	actSeq = {
		@MAct(actName=newObject, endStates={NewObject}),
		@MAct(actName=createObject, endStates={Created})      
	}
	),
	/* 7 */    
	@ANode(label="7:CustOrder",zone="6:Delivery", refCls=CustOrder.class, serviceCls=DataController.class, 
	actSeq={
		@MAct(actName=showObject),
		@MAct(actName=updateObject, endStates = {Updated})
	}),
	/* 8 */    
	@ANode(label="8:CollectPayment", zone="6:Delivery",refCls=CollectPayment.class, 
	serviceCls=DataController.class,
	nodeType=Coordinator,
	outNodes= {"10:Invoice"},
	actSeq={
		@MAct(actName=newObject, endStates={NewObject}),
		@MAct(actName=setDataFieldValues, attribNames = {"receivedOrder"}),
		@MAct(actName=createObject, endStates={Created})
	}),
	/* 9 */    
	@ANode(label="9:ShipOrder", zone="6:Delivery",refCls=ShipOrder.class, serviceCls=DataController.class,
	nodeType=Coordinator,
	outNodes= {"12:Shipment"},
	actSeq={
		@MAct(actName=newObject, endStates={NewObject}),
		@MAct(actName=setDataFieldValues, attribNames = {"receivedOrder"}, endStates={Created})
	}),
	/* 10 */    
	@ANode(label="10:Invoice", zone="8:CollectPayment",refCls=Invoice.class, serviceCls=DataController.class, 
	outNodes = {"11:AcceptPayment"},
	actSeq={
		// create new and wait until a new object is created
		@MAct(actName=newObject, endStates={NewObject}),
		@MAct(actName=setDataFieldValues, attribNames = {"order"}, endStates={Created})
	}),
	/* 11 */    
	@ANode(label="11:AcceptPayment", zone="8:CollectPayment",refCls=AcceptPayment.class, serviceCls=DataController.class,
	outNodes= {"14:Payment"},
	nodeType=Coordinator,      
	actSeq={
		@MAct(actName=newObject, endStates={NewObject}),
		@MAct(actName=setDataFieldValues, attribNames = {"invoice"}),
		@MAct(actName=createObject, endStates={Created})
	}),
	/* 12 */    
	@ANode(label="12:Shipment", zone="9:ShipOrder",refCls=Shipment.class, serviceCls=DataController.class,
	outNodes = { "13:CompleteOrder" },
	actSeq={
		// create new and wait until a new object is created
		@MAct(actName=newObject, endStates={NewObject}),
		@MAct(actName=setDataFieldValues, attribNames = {"order"}, endStates={Created})
	}),
	/* 13 */    
	@ANode(label="13:CompleteOrder",refCls=CompleteOrder.class, nodeType=Join, 
	outNodes= {"4:EndOrder"}),
	/* 14 */    
	@ANode(label="14:Payment", zone="11:AcceptPayment",refCls=Payment.class, serviceCls=DataController.class, 
	outNodes= { "15:CustOrder" },
	actSeq={
		@MAct(actName=newObject, endStates = {NewObject}),
		@MAct(actName=setDataFieldValues, attribNames = {"invoice"}, endStates = {Created}),
	}),
	/* 15 */    
	@ANode(label="15:CustOrder", zone="11:AcceptPayment",refCls=CustOrder.class,serviceCls=DataController.class,
	outNodes= { "13:CompleteOrder" },
	actSeq={
		@MAct(actName=filterInput, filterType=FilterCustOrderFromPayment.class),
		@MAct(actName=showObject, endStates = {Updated}),
	}),
})
/**END: activity graph configuration */
@DClass(serialisable=false, singleton=true)
public class HandleOrder {
	@DAttr(name = "id", id = true, auto = true, type = Type.Integer, length = 5, 
	optional = false, mutable = false)
	private int id;
	private static int idCounter = 0;
	
	// order 
	@DAttr(name="orders", type=Type.Collection,filter=@Select(clazz=CustOrder.class),serialisable=false)
	@DAssoc(ascName="create-order",role="mgmt",
	ascType=AssocType.One2Many,endType=AssocEndType.One,
	associate=@Associate(
	type=CustOrder.class,cardMin=0,cardMax=DCSLConstants.CARD_MORE,
	updateLink=false
	))
	private Collection<CustOrder> orders;
	// fill order
	@DAttr(name="fillOrders", type=Type.Collection,filter=@Select(clazz=FillOrder.class),serialisable=false)
	@DAssoc(ascName="fill-order",role="mgmt",
	ascType=AssocType.One2Many,endType=AssocEndType.One,
	associate=@Associate(
	type=FillOrder.class,cardMin=0,cardMax=DCSLConstants.CARD_MORE,
	updateLink=false
	))
	private Collection<FillOrder> fillOrders;
	// not used at the moment
	public HandleOrder(Integer id) {
		this.id = nextID(id);
	}	
	// for use by object form
	public HandleOrder() {
		this(null);
	}	
	public int getId() {
		return id;
	}	
	private static int nextID(Integer currID) {
		if (currID == null) { // generate one
			idCounter++;
			return idCounter;
		} else { // update
			int num;
			num = currID.intValue();
			
			if (num > idCounter) {
				idCounter=num;
			}   
			return currID;
		}
	}
}
\end{lstcodeplainssm}
%%%%%%%%%%%%%%%%%%%%%%%%%%%%%%%%%%%%%%
\begin{lstcodeplainssm}{The data class \clazz{Invoice} in Java}{lst:invoiceCls}
package org.jda.example.orderman.modules.invoice.model;
import java.util.Objects;
import org.jda.example.orderman.modules.customer.model.Customer;
import org.jda.example.orderman.modules.delivery.model.CollectPayment;
import org.jda.example.orderman.modules.order.model.CustOrder;
import org.jda.example.orderman.util.model.StatusCode;
import jda.modules.common.exceptions.ConstraintViolationException;
import jda.modules.common.types.Tuple;
import jda.modules.dcsl.syntax.DAttr;
import jda.modules.dcsl.syntax.DAttr.Type;
import jda.modules.dcsl.syntax.DOpt;

public class Invoice {
	@DAttr(name="id",type=Type.Integer,id=true,auto=true,mutable=false,optional=false,min=1)
	private int id;	
	private static int idCounter = 0;
	@DAttr(name="order", type=Type.Domain, mutable=false)
	private CustOrder order;
	// derived from order
	@DAttr(name="customer", type=Type.Domain, mutable=false, auto=true, serialisable=false)
	private Customer customer;
	@DAttr(name = "status", type = Type.Domain
	// not supported for Domain-typed attribute: auto=true
	, mutable=false
	)
	private StatusCode status;
	//  virtual link
	@DAttr(name="collectPayment",type=Type.Domain,serialisable=false,virtual=true)
	private CollectPayment collectPayment;
	public Invoice(Integer id, CustOrder order, StatusCode status) {
		this.id = nextID(id);
		this.order = order;
		this.status = status;
	}
	public Invoice(CustOrder order, StatusCode status) {
		this(null, order, status);
	}
	public StatusCode getStatus() {
		return status;
	}
	public boolean setStatus(StatusCode status) {
		this.status = status;	
		// update order as well
		order.invoiced();		
		return true;
	}	
	public int getId() {
		return id;
	}	
	private static int nextID(Integer currID) {
		if (currID == null) { // generate one
			idCounter++;
			return idCounter;
		} else { // update
			int num;
			num = currID.intValue();
			
			if (num > idCounter) {
				idCounter=num;
			}   
			return currID;
		}
	}
	public CustOrder getOrder() {
		return order;
	}
	public boolean setOrder(CustOrder order) {
		this.order = order;
		this.customer = order.getCustomer();
		return true;
	}
	public Customer getCustomer() {
		return customer;
	}	
	@DOpt(type = DOpt.Type.AutoAttributeValueSynchroniser)
	public static void synchWithSource(DAttr attrib, Tuple derivingValue, Object minVal, Object maxVal) throws ConstraintViolationException {
		String attribName = attrib.name();
		if (attribName.equals("id")) {
			int maxIdVal = (Integer) maxVal;
			if (maxIdVal > idCounter)
			idCounter = maxIdVal;
		}
	}	
	@Override
	public String toString() {
		return "Invoice (" + id + ", " + status + ")";
	}	
	@Override
	public int hashCode() {
		return Objects.hash(id);
	}
	@Override
	public boolean equals(Object obj) {
		if (this == obj)
		return true;
		if (obj == null)
		return false;
		if (getClass() != obj.getClass())
		return false;
		Invoice other = (Invoice) obj;
		return id == other.id;
	}	
}
\end{lstcodeplainssm}
\begin{lstcodeplainssm}{The control class \clazz{AcceptOrNot} in Java}{lst:acceptOrNotCls}
package org.jda.example.orderman.modules.handleorder.control.model;
import java.util.List;
import org.jda.example.orderman.modules.fillorder.model.FillOrder;
import org.jda.example.orderman.modules.order.model.CustOrder;
import jda.modules.common.exceptions.NotPossibleException;
import jda.modules.dcsl.syntax.DClass;
import jda.modules.mbsl.exceptions.DomainMessage;
import jda.modules.mbsl.model.graph.Edge;
import jda.modules.mbsl.model.graph.Node;
import jda.modules.mbsl.model.util.Decision;
/**
* @overview 
@DClass(serialisable=false)
public class AcceptOrNot implements Decision {
	@Override
	public Edge evaluate(Node decisionNode, Object[] args) throws NotPossibleException {
		CustOrder input = (CustOrder) args[0];	
		Class outCls;
		if (input.isRejected()) {
			outCls = EndOrder.class;
		} else {
			outCls = FillOrder.class;
		}		
		List<Edge> outEdges = decisionNode.getOut();
		for (Edge e : outEdges) {
			if (e.getTarget().getRefCls().equals(outCls)) {
				return e;
			}
		}		
		// no out edge is found: should not happen
		throw new NotPossibleException(DomainMessage.ERR_NO_SUITABLE_OUT_EDGE, new Object[] {decisionNode});
	}	
}
\end{lstcodeplainssm}

\begin{lstcodeplainssm}{The coordinator class \clazz{FillOrder} in Java}{lst:fillOrderCls}
package org.jda.example.orderman.modules.fillorder.model;
import java.util.Collection;
import java.util.Objects;
import org.jda.example.orderman.modules.delivery.model.Delivery;
import org.jda.example.orderman.modules.handleorder.model.HandleOrder;
import org.jda.example.orderman.modules.order.model.CustOrder;
import jda.modules.dcsl.syntax.DAssoc;
import jda.modules.dcsl.syntax.DAssoc.AssocEndType;
import jda.modules.dcsl.syntax.DAssoc.AssocType;
import jda.modules.dcsl.syntax.DAssoc.Associate;
import jda.modules.dcsl.syntax.DAttr;
import jda.modules.dcsl.syntax.DAttr.Type;
import jda.modules.dcsl.syntax.DCSLConstants;
import jda.modules.dcsl.syntax.DClass;
import jda.modules.dcsl.syntax.Select;

@DClass(serialisable=false)
public class FillOrder {
	@DAttr(name = "id", id = true, auto = true, type = Type.Integer, length = 5, 
	optional = false, mutable = false)
	private int id;
	private static int idCounter = 0;
	@DAttr(name="receivedOrder", type=Type.Domain, mutable=false,serialisable=false)
	private CustOrder receivedOrder;	
	// order 
	@DAttr(name="orders", type=Type.Collection,filter=@Select(clazz=CustOrder.class)
	,serialisable=false)
	@DAssoc(ascName="fill-order",role="mgmt",
	ascType=AssocType.One2Many,endType=AssocEndType.One,
	associate=@Associate(
	type=CustOrder.class,cardMin=0,cardMax=DCSLConstants.CARD_MORE,
	updateLink=false
	))
	private Collection<CustOrder> orders;	
	// delivery 
	@DAttr(name="deliveries", type=Type.Collection,filter=@Select(clazz=Delivery.class),serialisable=false)
	@DAssoc(ascName="delivery",role="mgmt",
	ascType=AssocType.One2Many,endType=AssocEndType.One,
	associate=@Associate(
	type=Delivery.class,cardMin=0,cardMax=DCSLConstants.CARD_MORE,
	updateLink=false
	))
	private Collection<Delivery> deliveries;	
	// virtual link to HandleOrder
	@DAttr(name="handleOrder",type=Type.Domain,serialisable=false,virtual=true)
	private HandleOrder handleOrder;	
	public FillOrder(Integer id, CustOrder receivedOrder) {
		this.id = nextID(id);
		this.receivedOrder = receivedOrder;
	}	
	public FillOrder(CustOrder receivedOrder) {
		this(null,receivedOrder);
	}
	public int getId() {
		return id;
	}
	public CustOrder getReceivedOrder() {
		return receivedOrder;
	}
	public void setReceivedOrder(CustOrder receivedOrder) {
		this.receivedOrder = receivedOrder;
	}
	@Override
	public String toString() {
		return "FillOrder (" + id + ")";
	}
	@Override
	public int hashCode() {
		return Objects.hash(id);
	}
	@Override
	public boolean equals(Object obj) {
		if (this == obj)
		return true;
		if (obj == null)
		return false;
		if (getClass() != obj.getClass())
		return false;
		FillOrder other = (FillOrder) obj;
		return id == other.id;
	}	
	private static int nextID(Integer currID) {
		if (currID == null) { // generate one
			idCounter++;
			return idCounter;
		} else { // update
			int num;
			num = currID.intValue();
			
			if (num > idCounter) {
				idCounter=num;
			}   
			return currID;
		}
	}
}
\end{lstcodeplainssm}	
	