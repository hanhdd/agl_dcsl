\section{Comparing to DDD frameworks} \label{apex:compareAGLwithOther}


Table~\ref{tab:Comparing-the-expressiveness} presents the evaluation table of \agldcsl and the existing DDD approaches (AL, XL and AD). 

\begin{table}[ht]
	\setlength\tabcolsep{1pt}
	\centering
	%\footnotesize
	\caption{Comparing the expressiveness of AGL to AL, XL}
	\label{tab:Comparing-the-expressiveness}
	\begin{tabular}{|l|l|l|}
		\hline
		\multicolumn{1}{|c|}{\textbf{AGL}} & \multicolumn{1}{c|}{\textbf{AL}}                                                            & \multicolumn{1}{c|}{\textbf{XL}}                                                       \\ \hline
		DClass                             & -                                                                                           & -                                                                                      \\ \hline
		Mutable                            & Property.editing                                                                            & -                                                                                      \\ \hline
		DAttr                              &             -                                                                                &                                                                                        \\ \hline
		Unique                             & -                                                                                        & -                                                                                      \\ \hline
		optional                           & \begin{tabular}[c]{@{}l@{}}jdo.Column.allowsNull,\\ (Property.optionality)\end{tabular} & Required                                                                               \\ \hline
		id                                 & jdo.PrimaryKey.value                                                                        & jpa.Id                                                                                 \\ \hline
		auto                               & -                                                                                           & -                                                                                  \\ \hline
		length                             & \begin{tabular}[c]{@{}l@{}}jdo.Column.length,\\  (Property.maxLength)\end{tabular}          & -                                                                 \\ \hline
		min                                & -                                                                                           & Min(v).value                                                                           \\ \hline
		max                                & -                                                                                           & Max(v).value                                                                           \\ \hline
		DAssoc                             &   -                                                                                          &    -                                                                                    \\ \hline
		ascName                            & -                                                                                           & \begin{tabular}[c]{@{}l@{}}jpa.OneToMany, jpa.ManyToOne,\\ jpa.ManyToMany\end{tabular} \\ \hline
		ascType                            & -                                                                                           & -                                                                                      \\ \hline
		role                               & -                                                                                           & -                                                                                      \\ \hline
		endType                            & -                                                                                           & -                                                                                      \\ \hline
		associate.type                     & -                                                                                           & -                                                                                      \\ \hline
		associate.cardMin                  & -                                                                                           & -                                                                                      \\ \hline
		associate.cardMax                  & -                                                                                           & -                                                                                      \\ \hline
		DOpt                               &     -                                                                                        &    -                                                                                    \\ \hline
		type                               & -                                                                                           & -                                                                                      \\ \hline
		requires                           & -                                                                                           & -                                                                                      \\ \hline
		effects                            & -                                                                                           & -                                                                                      \\ \hline
		AttrRef                            & -                                                                                           & -                                                                                      \\ \hline
		value                              & -                                                                                           & -                                                                                      \\ \hline
		AGraph                             & -                                                                                           & -                                                                                      \\ \hline
		ANode                              & -                                                                                           & -                                                                                      \\ \hline
		MAct                               & 
		\begin{tabular}[c]{@{}l@{}}	defined by the controllers:\\only CRUD and reporting
			
		\end{tabular} 	
		
		&\begin{tabular}[c]{@{}l@{}} direct mapping from the domain\\ object model into the UI                    \end{tabular} 	                                                                                  \\ \hline
	\end{tabular}
\end{table}

In the Table~\ref{tab:eval_expr_agl} points out that \agldcsl is more expressive than AL and XL in both structural and behavioral aspects. These two languages only partially support structural aspect and they do not support behavioral aspect. In particular, AL and XL's support for Associative Field is very limited compared to \agldcsl. AD's expressiveness of domain behaviors is better than \agl, but again the behavior model lacks an explicit connection to structural models.
In the our previous work~\cite{le_domain_2018}, the five meta-attributes are \clazz{DClass}, \clazz{DAttr}, \clazz{DAssoc}, \clazz{DOpt} and \clazz{AttrRef} are eight properties: 
\begin{enumerate}
	\item the domain field~\clazz{mutable}: whether or not the objects of a class are mutable  (i.e. its value can be changed),
	\item the domain field~\clazz{unique}: whether or not the values of a field are unique,
	\item the domain field~\clazz{optional}: whether or not a field is optional (i.e. its value needs not be initialized when an object is created),
	\item the domain field~\clazz{id} is an identifier field: whether or not a field is an object id field,
	\item the domain field~\clazz{auto}: whether or not the values of a field are automatically generated (by the system),
	\item the domain field~\clazz{length}: the maximum length (if applicable) of a field (i.e. the field’s values must not exceed this length),
	\item the domain field~\clazz{max} or~\clazz{min}:  The maximum value (if applicable) of a field (i.e. the field’s values must not be higher than this) or the min value (if applicable) of a field (i.e. the field’s values must not be lower than it) and
	\item the domain field~\clazz{ascName}:  which specifies the association name.
		
\end{enumerate}

The ratio 4/8 for AL \wrt the term Domain Field means that AL only supports 4 out of the total of 8 properties of the meta-attribute \clazz{DAttr} (used in Domain Field). 

Note that the ratio aims to evaluate the current works from structural aspect, thus, this evaluation step coincides with the one for \dcsl, reported in our previous work~\cite{le_domain_2018}. %

\textit{Example}
\begin{enumerate}
	\item Class Payment is immutable, i.e. all objects of this class are immutable~\attrib{DClass}{mutable=true}.
\item Field Student.name is mutable, i.e. its value can be changed~\attrib{DAttr}{mutable=true}.
\item Student.name is not optional, i.e. its value must be specified for each object~\attrib{DAttr}{optional=false}.
\item Student.name does not exceed 30 characters in length~\attrib{DAttr}{length=30}.
\item Student.name is not unique, i.e. its value may duplicate among objects~\attrib{DAttr}{unique=false}.
\item Student.id is an id field~\attrib{DAttr}{id=true}.
\item Student.id is auto, i.e. its value is automatically generated~\attrib{DAttr}{auto=true}.
\item The minimum value of CourseModule.semester is 1~\attrib{DAttr}{min=1}.
	\item The maximum value of CourseModule.semester is 8~\attrib{DAttr}{max=8}.
	\item The association constraint, e.g. with respect to the enrols-in association, a Student is enrolled in zero or no more than 30 CourseModules,
	and a CourseModule is enrolled in by zero or more Students~\attrib{DAssoc}{Associate(type=Enrolment.class, cardMin=0, cardMax=30)}.
\end{enumerate}