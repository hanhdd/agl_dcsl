\section{Comparing to DDD frameworks} \label{apex:compareAGLwithOther}


Table~\ref{tab:Comparing-the-expressiveness} presents the evaluation table of \agldcsl and the existing DDD approaches (AL, XL and AD). 

% Please add the following required packages to your document preamble:
% \usepackage[table,xcdraw]{xcolor}
% If you use beamer only pass "xcolor=table" option, i.e. \documentclass[xcolor=table]{beamer}
%%%%%%%%%%%%%%%%%%%%%%%%%%

\begin{minipage}{\textwidth}
	\centering
	\caption{Comparing the expressiveness of AGL to AL, XL} \label{tab:Comparing-the-expressiveness}
	
	\begin{tabular}{|rccl|}
		\hline
		\multicolumn{1}{|c|}{\textbf{AGL}}      & \multicolumn{1}{c|}{\textbf{AL}}                                                                                   & \multicolumn{1}{c|}{\textbf{XL}}                                                                                       & \textbf{Essential properties} \\ \hline
		\multicolumn{4}{|l|}{\cellcolor[HTML]{DAE8FC}DClass}                                                                                                                                                                                                                                                                  \\ \hline
		\multicolumn{1}{|r|}{Mutable}           & \multicolumn{1}{c|}{DomainObject.editing}                                                                          & \multicolumn{1}{c|}{-}                                                                                                 & \multicolumn{1}{c|}{P1}       \\ \hline
		\multicolumn{4}{|l|}{\cellcolor[HTML]{DAE8FC}DAttr}                                                                                                                                                                                                                                                                   \\ \hline
		\multicolumn{1}{|r|}{unique}            & \multicolumn{1}{c|}{-$^(\footnote{AL supports property Property.mustSatisfy which may be used to implement the constraint.}$^)}                                                                                         & \multicolumn{1}{c|}{-}                                                                                                 & \multicolumn{1}{c|}{P2}       \\ \hline
		\multicolumn{1}{|r|}{optional}          & \multicolumn{1}{c|}{\begin{tabular}[c]{@{}c@{}}jdo$^(\footnote{Java Data Objects (JDO)~\cite{JDO_Annotations_Reference2021} }$^).Column.allowsNull,\\ (Property.optionality)\end{tabular}}   & \multicolumn{1}{c|}{Required}                                                                                          & \multicolumn{1}{c|}{P3}       \\ \hline
		\multicolumn{1}{|r|}{mutable}           & \multicolumn{1}{c|}{Property.editing}                                                                              & \multicolumn{1}{c|}{-}                                                                                                 & \multicolumn{1}{c|}{P1}       \\ \hline
		\multicolumn{1}{|r|}{id}                & \multicolumn{1}{c|}{jdo.PrimaryKey.value}                                                                          & \multicolumn{1}{c|}{jpa.Id}                                                                                            & \multicolumn{1}{c|}{P4}       \\ \hline
		\multicolumn{1}{|r|}{auto}              & \multicolumn{1}{c|}{-}                                                                                             & \multicolumn{1}{c|}{-$^(\footnote{XL supports property ha.Formula that may be use to implement formula for value generation function.}$^)}                                                                                            & \multicolumn{1}{c|}{P5}       \\ \hline
		\multicolumn{1}{|r|}{length}            & \multicolumn{1}{c|}{\begin{tabular}[c]{@{}c@{}}jdo.Column.length,\\ (Property.maxLength)\end{tabular}}             & \multicolumn{1}{c|}{jpa$^(\footnote{Java Persistence API (JPA)~\cite{Java_Persistence_API2013}}$^).Column.length}                                                                             & \multicolumn{1}{c|}{P6}       \\ \hline
		\multicolumn{1}{|r|}{min}               & \multicolumn{1}{c|}{-}                                                                                             & \multicolumn{1}{c|}{Min$^(\footnote{\label{bv}Bean Validator (BV)~\cite{Jakarta_Bean_Validation}}$^).value}                                                                                      & \multicolumn{1}{c|}{P7}       \\ \hline
		\multicolumn{1}{|r|}{max}               & \multicolumn{1}{c|}{-}                                                                                             & \multicolumn{1}{c|}{Max$^($^\ref{bv}$^).value}                                                                                      & \multicolumn{1}{c|}{P8}       \\ \hline
		\multicolumn{4}{|l|}{\cellcolor[HTML]{DAE8FC}DAssoc}                                                                                                                                                                                                                                                                  \\ \hline
		\multicolumn{1}{|r|}{ascName}           & \multicolumn{1}{c|}{-}                                                                                             & \multicolumn{1}{c|}{-}                                                                                                 &                               \\ \hline
		\multicolumn{1}{|r|}{ascType}           & \multicolumn{1}{c|}{-}                                                                                             & \multicolumn{1}{c|}{\begin{tabular}[c]{@{}c@{}}jpa.OneToMany,\\ jpa.ManyToOne,\\ jpa.ManyToMany\end{tabular}}          &                               \\ \hline
		\multicolumn{1}{|r|}{role}              & \multicolumn{1}{c|}{-}                                                                                             & \multicolumn{1}{c|}{-}                                                                                                 &                               \\ \hline
		\multicolumn{1}{|r|}{endType}           & \multicolumn{1}{c|}{-}                                                                                             & \multicolumn{1}{c|}{-}                                                                                                 &                               \\ \hline
		\multicolumn{1}{|r|}{associate.type}    & \multicolumn{1}{c|}{-}                                                                                             & \multicolumn{1}{c|}{-}                                                                                                 &                               \\ \hline
		\multicolumn{1}{|r|}{associate.cardMin} & \multicolumn{1}{c|}{-}                                                                                             & \multicolumn{1}{c|}{-}                                                                                                 &                               \\ \hline
		\multicolumn{1}{|r|}{associate.cardMax} & \multicolumn{1}{c|}{-}                                                                                             & \multicolumn{1}{c|}{-}                                                                                                 &                               \\ \hline
		\multicolumn{4}{|l|}{\cellcolor[HTML]{DAE8FC}DOpt}                                                                                                                                                                                                                                                                    \\ \hline
		\multicolumn{1}{|r|}{type}              & \multicolumn{1}{c|}{-}                                                                                             & \multicolumn{1}{c|}{-}                                                                                                 &                               \\ \hline
		\multicolumn{1}{|r|}{requires}          & \multicolumn{1}{c|}{-}                                                                                             & \multicolumn{1}{c|}{-}                                                                                                 &                               \\ \hline
		\multicolumn{1}{|r|}{effects}           & \multicolumn{1}{c|}{-}                                                                                             & \multicolumn{1}{c|}{-}                                                                                                 &                               \\ \hline
		\multicolumn{4}{|l|}{\cellcolor[HTML]{DAE8FC}AttrRef}                                                                                                                                                                                                                                                                 \\ \hline
		\multicolumn{1}{|c|}{value}             & \multicolumn{1}{c|}{-}                                                                                             & \multicolumn{1}{c|}{-}                                                                                                 &                               \\ \hline
	
	\end{tabular}
	
\end{minipage}
	
	In the Table~\ref{tab:eval_expr_agl} points out that \agldcsl is more expressive than AL and XL in both structural and behavioral aspects. These two languages only partially support structural aspect and they do not support behavioral aspect. In particular, AL and XL's support for Associative Field is very limited compared to \agldcsl. AD's expressiveness of domain behaviors is better than \agl, but again the behavior model lacks an explicit connection to structural models.
	
	In the our previous work~\cite{le_domain_2018}, the five meta-attributes are \clazz{DClass}, \clazz{DAttr}, \clazz{DAssoc}, \clazz{DOpt} and \clazz{AttrRef} are eight  essential properties: 
	\begin{enumerate}
		\item[P1.] the domain field~\clazz{mutable}: whether or not the objects of a class are mutable  (i.e. its value can be changed),
		\item[P2.] the domain field~\clazz{unique}: whether or not the values of a field are unique,
		\item[P3.] the domain field~\clazz{optional}: whether or not a field is optional (i.e. its value needs not be initialized when an object is created),
		\item[p4.] the domain field~\clazz{id} is an identifier field: whether or not a field is an object id field,
		\item[P5.] the domain field~\clazz{auto}: whether or not the values of a field are automatically generated (by the system),
		\item[P6.] the domain field~\clazz{length}: the maximum length (if applicable) of a field (i.e. the field’s values must not exceed this length),
		\item[P7.] the domain field~\clazz{min}:  the min value (if applicable) of a field (i.e. the field’s values must not be lower than it) and
		\item[P8.] the domain field~\clazz{max}:  the maximum value (if applicable) of a field (i.e. the field’s values must not be higher than this).
				
	\end{enumerate}
	%%%%%%%%%%%%%%%%%%%%%%%%%%%%%%%%
	The ratio 4/8 for AL \wrt the term Domain Field means that AL only supports 4 out of the total of 8 properties of the meta-attribute \clazz{DAttr} (used in Domain Field). 
	
	Note that the ratio aims to evaluate the current works from structural aspect, thus, this evaluation step coincides with the one for \dcsl, reported in our previous work~\cite{le_domain_2018}. %
	%%%%%%%%%%%%%%%%%%%%%%%%%%%%%%%%%%%%%%
	\textit{Example}
	\begin{enumerate}
		\item Class Payment is immutable, i.e. all objects of this class are immutable~\attrib{DClass}{mutable=true}.
		\item Field Student.name is mutable, i.e. its value can be changed~\attrib{DAttr}{mutable=true}.
		\item Student.name is not optional, i.e. its value must be specified for each object~\attrib{DAttr}{optional=false}.
		\item Student.name does not exceed 30 characters in length~\attrib{DAttr}{length=30}.
		\item Student.name is not unique, i.e. its value may duplicate among objects~\attrib{DAttr}{unique=false}.
		\item Student.id is an id field~\attrib{DAttr}{id=true}.
		\item Student.id is auto, i.e. its value is automatically generated~\attrib{DAttr}{auto=true}.
		\item The minimum value of CourseModule.semester is 1~\attrib{DAttr}{min=1}.
		\item The maximum value of CourseModule.semester is 8~\attrib{DAttr}{max=8}.
		\item The association constraint, e.g. with respect to the enrols-in association, a Student is enrolled in zero or no more than 30 CourseModules,
		and a CourseModule is enrolled in by zero or more Students~\attrib{DAssoc}{Associate(type=Enrolment.class, cardMin=0, cardMax=30)}.
	\end{enumerate}