\section{Introduction}
\label{sect:introduction}

The object-oriented domain-driven design (DDD)~\cite{evans_domain-driven_2004} is aimed at developing software in an iterative manner around a realistic model of the problem domain, which captures the domain requirements and is technically feasible for implementation. To achieve this, a close collaboration among all stakeholders, including domain experts, end-users, and developers, is required. The ubiquitous language~\cite{evans_domain-driven_2004} is used to construct the domain model and create an object-oriented implementation of this model. The DDD method employs a conceptual layered software architecture, which places the domain model at the core layer and other architectural concerns, such as user interface, persistence, etc., in other layers surrounding this core.
%
% Why to integrate behavioral aspects???
Recent research in DDD~\cite{dan_haywood_apache_2013, paniza_learn_2011} proposes annotation-based domain-specific languages (aDSLs), which are written inside a host object-oriented programming language (OOPL), to facilitate the construction of domain models. A straightforward way to obtain an executable version of the software from such a representation of the domain model is to directly embed the implementation in the OOPL along with other concerns for the entire program. Another indirect way is to follow model-driven approaches by composing the domain model with other concerns expressed at a high level using a general language like UML and DSLs. The final program can then be obtained by model transformations, either model-to-model or model-to-text. This work focuses on an alternative approach to achieve this goal, which is a refinement of an aDSL-based software development method for DDD that we proposed in a recent work~\cite{le_domain_2018}.

We aim to develop an extension of the domain model that can capture behavioral aspects of the domain, resulting in a unified domain model that facilitates software construction. However, modeling behavioral aspects within this unified domain model can be challenging. To overcome this challenge, we propose a language with specific support for incorporating domain behaviors. This language aims to bridge the gap between the domain model and its implementation, making it easier to build software through model transformations from a domain model that incorporates behavioral models expressed in UML and DSLs.

%Specifically, we define a novel aDSL, named \agl~(Activity Graph Language) with two main aims: (1)~to represent behavioral aspects (that could be captured using UML Activity diagrams and Statecharts~\cite{omg_unified_2017}) and (2)~to incorporate them as part of the unified domain model. %
%%How to incorporate the \agl~specification with the remaining part of the unified domain model???
%For the first aim, we scope \agl~around a restricted domain of the UML activity graph language that is defined based on essential UML activity modeling patterns~\cite{omg_unified_2017}. We adopt the meta-modeling approach for DSLs~\cite{kleppe_software_2008} and use UML/OCL~\cite{omg_unified_2017, omg_object_2014} to specify the abstract and concrete syntax models of \agl. %
%%In particular, we propose a compact annotation-based concrete syntax model that includes few concepts. We systematically develop this syntax using a transformation from the abstract syntax model, which is a conceptual model of the activity graph domain. We then define a formal semantics for \agl. 
%For the second aim, we employ our previously-developed aDSL, named \dcsl, in order to express a so-called unified class model. The unified class model is viewed as an extended domain model in MOSA (a module-based software architecture~\cite{le_generative_2018} that we have recently developed for DDD). 
%This model includes new domain classes, referred to as \textit{activity classes}, that are attached with \agl's activity graph: Each activity class corresponds to an executable node of \agl's activity graph, that performs a set of core actions on the software modules in MOSA. These actions concern the manipulation of instances of the domain class (owned by the corresponding software module). The composition of the unified class model and the AGL specifications results in a unified domain model. We demonstrate our method with an implementation in \jdomainapp~and evaluate \agl~to show that it is essentially expressive and usable for designing real-world software.
%%As explained above, a key benefit of combining \dcsl~and \agl~in MOSA is that it helps define a complete executable model for the software. Further, this software is automatically generated using a Java software framework, named \jdomainapp~\cite{le_jdomainapp_2017}, that we have developed.

The proposed language, AGL~(Activity Graph Language), has two main aims: (1)~to represent behavioral aspects that could be captured using UML Activity diagrams and Statecharts~\cite{omg_unified_2017}, and (2)~to incorporate these aspects as part of the unified domain model. AGL is scoped around a restricted domain of the UML activity language that is based on essential UML activity modeling patterns~\cite[p. 373 ]{omg_unified_2017}. The meta-modeling approach for DSLs~\cite{kleppe_software_2008} is used, with UML/OCL~\cite{omg_unified_2017, omg_object_2014} to specify the abstract and concrete syntax models of AGL. 
%
To incorporate AGL into the unified domain model, the previously-developed aDSL, \dcsl~\cite{le_domain_2018}, is used to express a \textit{unified class model}. The unified class model is viewed as an extended domain model in MOSA~(a module-based software architecture recently developed for DDD~\cite{le_generative_2018}) and includes new domain classes, referred to as \textit{activity classes}, that are attached with \agl's activity graph. Each activity class corresponds to an executable node of \agl's activity graph, which performs a set of core actions on the software modules in MOSA.
%
The composition of the unified class model and the AGL specifications results in a \textit{unified domain model}, which is essentially expressive and usable for designing real-world software. The method is implemented in jDomainApp and evaluated to demonstrate its effectiveness.

In brief, our paper makes the following contributions:
%
\begin{itemize}[leftmargin=*]
	\item A mechanism to incorporate behavioral aspects for a unified domain model: An aDSL (named \agl) is defined to represent the domain behaviors for the incorporation;
	%\item define a set of essential module actions for the software modules of MOSA and a set of patterns to capture domain behaviors;	
	\item A unified modeling method for domain-driven software development;
	\item An implementation in the \jdomainapp~framework for the proposed method; and
	\item An evaluation of \agl~to show that it is essentially expressive and usable for designing real-world software
\end{itemize}

The rest of the paper is structured as follows. Section~\ref{sect:background} presents our motivating example and the technical background. 
Section~\ref{sect:overviewApproach} overviews our approach to incorporating behavioral aspects into a domain model. 
Section~\ref{sect:actSemantics} provides formal semantics for module actions. Section~\ref{sect:behaviorPatterns} explains the patterns to capture domain behaviors.  
Section~\ref{sect:agl} specifies \agl. 
Section~\ref{sect:caseStudyToolSupport} discusses the case study \orderman and tool support.
An evaluation of \agl~is presented in Section~\ref{sect:evaluation}. Section~\ref{sect:threats} discusses threats to the validity of our work.
Section~\ref{sect:relatedwork} reviews the related work. This paper closes with a conclusion and an outlook on future work.
